\documentclass[12pt]{article}

\begin{document}

\section{Introduction}
	- "models depict biological processes in simplified and general ways that provide insight into factors that are responsible for observed patterns" (johnson and omland 2004)
\section{Materials and Methods}
	- use AIC and BIC (And R sqaured) so can compare them
	- briere: used reference to give estimation of starting value for $b_0$ (from briere et al 1999)

\section{Results}
- shape of curves very different - many possibilities
- some flat = temperature range not sufficient to see decline in rate??
- only increasing/decreasing = again could be that temp range not sufficient - MAYBE DO SHCOFIELD UPPER AND LOWER EQUATIONS TO TEST THIS?

\section{Discussion}

- model selected ultimately depends on the models chosen for the candidate set so if you dont include a model that could best represent the process the could lead to misguided inference (=pitfall)(johnson and omland 2004)

-predictions and parameter estimates must be biologically plausible - check! (johnson + omland 2004)

Validation - the validation of a model isn't to work out whether its 'true', as a model is not a hypothesis and isn't directly verifiable by experiment. Validation is about ensuring the model generates a good testable hypothesis relevant to important problems. (levins 1966)

Unlike the theory, models are constrained to a few components at a time, even if the theory is complex. Therefore, a good theory is usually a cluster of models. (levins 1966)

-briere model doesnt cope if temperature doesnt drop either side of an optimum value

-briere model allows representation of asymmetry around optimum temperature (whereas quad and cubic dont??)
-many models dont include minimum and maximum viable temperatures (t0 and tm) - briere does, schoolfield doesnt (see table 4, briere et al 1999)

\end{document}